\begin{enumerate}
    \item Mengembangkan  perangkat lunak \textit{traffic aware routing engine} berbasis Time-Dependent Customizable Route Planning (TDCRP) yang mampu menghasilkan rute terpendek dan rute alternatif dengan mempertimbangkan variasi kecepatan lalu lintas.
    \item Membandingkan performa algoritma Time-independent CRP dan Time-dependent CRP dalam hal waktu kustomisasi, waktu kueri, serta kualitas rute yang dihasilkan.
    \item Mengembangkan fitur \textit{online map-matching} yang mampu menghasilkan prediksi trajektori pada ruas jalan jaringan jalan berdasarkan titik-titik koordinat GPS sebelumnya secara \textit{real-time}.
    \item  Membangun dan menerapkan model Multivariate Time-Series Graph Neural Network (MTGNN) untuk memodelkan dan memprediksi kecepatan lalu lintas berdasarkan data historis lalu lintas.
    \item Mengintegrasikan prediksi kecepatan lalu lintas dari GNN ke dalam fungsi biaya pada TDCRP, sehingga perangkat lunak \textit{routing engine} mampu menghasilkan rute yang lebih optimal.  
    \item Menyediakan prototipe perangkat lunak sumber terbuka \textit{traffic aware routing engine} yang dapat dijalankan secara interaktif, berbasis bahasa pemrograman Go (\textit{routing engine backend}), Python (\textit{machine Learning}), dan TypeScript (\textit{frontend}), untuk mendemonstrasikan hasil penelitian.
\end{enumerate}