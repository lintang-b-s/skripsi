\begin{enumerate}
    \item Mengembangkan  perangkat lunak \textit{traffic aware routing engine} berbasis Time-Dependent Customizable Route Planning (TDCRP) yang mampu menghasilkan rute terpendek dan rute alternatif dengan mempertimbangkan variasi kondisi lalu lintas.
    \item Membandingkan performa algoritma Time-Independent CRP, Contraction Hierarchies (CH), dan Time-Dependent CRP dalam hal waktu prapemrosesan, waktu kueri, serta kualitas rute yang dihasilkan.
    \item Mengembangkan fitur \textit{online map-matching} yang mampu menghasilkan      prediksi trajektori pada ruas jalan jaringan jalan berdasarkan trajektori GPS secara \textit{real time}.
    \item  Membangun dan menerapkan model Graph Neural Network (Multivariate Time-Series Graph Neural Network) untuk memodelkan dan memprediksi kondisi lalu lintas berdasarkan data historis lalu lintas.
    \item Mengintegrasikan prediksi kondisi lalu lintas dari GNN ke dalam fungsi biaya pada TDCRP, sehingga mesin perutean mampu menghasilkan rute yang lebih optimal secara kontekstual.  
    
    \item Menyediakan prototipe perangkat lunak sumber terbuka \textit{traffic aware routing engine} yang dapat dijalankan secara interaktif, berbasis bahasa pemrograman Go (backend), Python (Machine Learning), dan TypeScript (frontend), untuk mendemonstrasikan hasil penelitian.
\end{enumerate}