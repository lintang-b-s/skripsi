Penelitian pengembangan perangkat lunak \textit{traffic aware routing engine} pada \textit{openstreetmap} dilakukan untuk menghasilkan rute yang lebih optimal dan kontekstual dibandingkan dengan \textit{time-independent routing engine} yang selama ini banyak digunakan pada perangkat lunak sumber terbuka. Manfaat praktis dari penelitian ini adalah tersedianya prototype perangkat lunak \textit{routing engine} berbasis sumber terbuka yang mampu melakukan perhitungan rute secara efisien sekaligus mempertimbangkan variasi lalu lintas ataupun prediksi lalu lintas ke depan, sehingga dapat diaplikasikan dalam sistem navigasi dan manajemen transportasi cerdas. Selain itu, fitur \textit{online map-matching} yang dikembangkan mendukung penentuan posisi pengguna secara real-time dengan tingkat akurasi yang tinggi yang penting bagi aplikasi navigasi. 