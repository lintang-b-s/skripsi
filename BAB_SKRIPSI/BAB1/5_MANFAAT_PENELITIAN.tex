Penelitian pengembangan perangkat lunak \textit{traffic aware routing engine} pada \textit{openstreetmap} dilakukan tersedianya perangkat lunak sumber terbuka yang dapat menghasilkan rute yang lebih optimal dibandingkan dengan \textit{time-independent routing engine} yang selama ini banyak digunakan pada perangkat lunak sumber terbuka. Manfaat dari penelitian ini adalah tersedianya \textit{prototype} perangkat lunak \textit{routing engine} sumber terbuka yang mampu melakukan perhitungan rute dengan cepat sekaligus mempertimbangkan data historis lalu lintas ataupun prediksi lalu lintas ke depan, sehingga dapat diaplikasikan dalam sistem navigasi dan manajemen transportasi cerdas. Selain itu, fitur \textit{online map-matching} yang dikembangkan mendukung penentuan posisi pengguna secara \textit{real-time} dengan tingkat akurasi yang tinggi yang penting bagi aplikasi navigasi. 