Algoritma Komputer saat ini digunakan dalam berbagai bidang untuk meningkatkan efektifitas pekerjaan dan taraf hidup manusia. Salah satu bidang yang saat ini sedang berkembang dengan pesat adalah \textit{route planning} dengan menggunakan modifikasi dari algoritma dijkstra untuk mencari rute optimal dengan cepat. 

Menghitung rute optimal di dalam jaringan transportasi memiliki banyak aplikasi di dunia nyata, kita sering menggunakan fungsi ini untuk merencakanan perjalanan dengan kendaraan, terdapat juga aplikasi lain seerti perencanaan logistik dan transportasi publik.

Meskipun algoritma klasik dijkstra (\cite{Dijkstra59}) berjalan dalam waktu yang hampir linier, penerapannya pada graph berukuran benua memerlukan beberapa detik sehingga tidak bisa digunakan untuk aplikasi \textit{real-time}. Hal ini memotivasi banyak penelitian (\cite{Delling2009} dan \cite{Bast2015}) tentang metode akselerasi yang membagi pekerjaan perencanaan rute menjadi dua tahap: langkah \textit{preprocessing}, yang mungkin memakan waktu beberapa menit atau bahkan berjam-jam untuk menhasilkan kumpulan data tambahan yang dapat mempercepat proses kueri, tahap  kedua berupa fase kueri yang dapat dijawab dalam hitungan milidetik. Sebagian besar penelitian sebelumnya telah diuji pada jaringan jalan di Eropa dan Amerika Serikat menggunakan waktu tempuh sebagai kriteria optimasi utama.

Terdapat juga beberapa metode yang dapat mempertimbangkan histori kondisi lalu lintas dan keadaan lalu lintas terkini (\cite{DellingTD2009}, \cite{Baum2016}, dan \cite{Veit2013}) yang mana mengganti bobot dari \textit{edge} dengan menggunakan \textit{travel time function} $f : \mathbb{R} \to \mathbb{R}_{\geq 0}$, lalu menggunakan \textit{time-dependent dijkstra} dan \textit{profile search} untuk melakukan kueri dan \textit{preprocessing}. Performa dari berbagai metode \textit{time-dependent route planning} dapat menyamai performa dari \textit{time-independent route planning} di berbagai data jaringan jalan. Salah satu penelitian di bidang ini adalah penelitian oleh \cite{Delling2015} yang menggunakan metode \textit{customizable route planning}, dimana metode ini dibagi menjadi 3 tahap, yaitu praproses, kustomisasi, dan kueri. Praproses mempartisi graf jaringan jalan menjadi beberapa partisi, kustomisasi menghitung rute terpendek dari semua pasangan simpul yang berada pada batas partisi, dan kueri menjalankan kueri dalam hitungan beberapa milidetik. Kustomisasi pada metode \textit{Customizable Route Planning} dapat dilakukan hanya dalam hitungan beberapa detik, yang memungkinkan perbaruan data graf menyesuaikan kondisi lalu lintas. Penelitian oleh \cite{Baum2016} melakukan ekstensi dari metode \textit{Customizable Route Planning} dimana kueri dapat mempertimbangkan kondisi trafik pada masa mendatang.

Dalam konteks aplikasi \textit{routing engine}, \textit{map matching} memiliki peranan yang sangat penting karena kualitas hasil perencanaan rute sangat bergantung pada akurasi posisi pengguna pada jaringan jalan. \textit{Map matching} adalah proses mencocokkan titik-titik lokasi hasil pelacakan GPS dengan ruas jalan yang sebenarnya pada graf jaringan transportasi (\cite{Newson2009}). Data GPS yang mentah sering kali mengandung error, seperti deviasi posisi akibat gangguan sinyal satelit, sehingga posisi pengguna tidak selalu tepat berada di atas ruas jalan. Tanpa proses \textit{map matching}, sistem rute dapat menghasilkan jalur yang keliru, misalnya menempatkan kendaraan di jalan paralel yang berbeda atau memberikan instruksi navigasi dan pengalihan rute yang tidak relevan.


Banyak penelitian mengenai \textit{map matching} yang sudah dilakukan sebelumnya. Salah satu metode \textit{map matching} yang paling sering digunakan adalah Hidden Markov Model yang diperkenalkan pada penelitian oleh \cite{Newson2009}, dimana setiap trajektori titik gps diperlakukan sebagai observasi, sedangkan posisi sbeenarnya pada ruas jalan dianggap sebagai \textit{hidden states}. Dengan memanfaatkan probabilitas emisi (kedekatan titik GPS dengan kandidat jalan) dan probabilitas transisi (kemungkinan berpindah antar ruas jalan), algoritma viterbi dapat digunakan untuk menemukan jalur optimal yang sesuai dengan trajektori GPS.

Selain metode Hidden Markov Model yang mana hanya dapat melakukan \textit{map matching} secara \textit{offline}, terdapat juga pendekatan \textit{Candidate-Evolving Model} yang mempertahankan sekumpulan kandidat jalur dan memperbaruinya seiring dengan bertambahnya titik GPS baru. Salah satu metode representatif dari kategori ini adalah \textit{Multiple Hypothesis Technique (MHT)} yang diperkenalkan pada penelitian oleh \cite{Taguchi2019}. MHT berfokus pada pemeliharaan daftar hipotesis kandidat jalan dengan biaya komputasi yang lebih rendah. Setiap hipotesis dievaluasi menggunakan fungsi skor yang mempertimbangkan kesesuaian antara titik GPS dan kandidat jalan terdekat, kemudian daftar hipotesis diperbarui untuk memastikan kemungkinan jalur yang benar tetap terjaga. Keunggulan dari metode ini adalah kemampuannya untuk diadaptasikan dalam skenario \textit{online map matching}, termasuk melakukan prediksi jalur ke depan berdasarkan hipotesis yang ada. 

Terdapat juga beberapa perangkat lunak sumber terbuka untuk perencanaan rute pada openstreetmap. Perangkat lunak Open Source Routing Machine (OSRM)  (\cite{Luxen2011}) menggunakan pendekatan yang mirip dengan Customizable Route Planning (\cite{Delling2015}) dan juga mampu melakukan kueri dalam hitungan beberapa milidetik dan mampu melakukan kustomisasi dalam hitungan beberapa detik.



Mayoritas perangkat lunak sumber terbuka untuk \textit{route planning} pada Openstreetmap berfokus pada \textit{time-independent route planning} dan tidak mengintegrasikannya dengan model \textit{traffic speed forecasting} sehingga hasil kueri dari perangkat lunak ini tidak mempertimbangkan kondisi lalu lintas pada masa depan. Juga pada perangkat lunak ini, tidak terdapat fitur \textit{online map matching}, yaitu menentukan lokasi yang sebenarnya pada ruas jalan pada jaringan jalan dari data trajektori GPS secara \textit{real-time}.

Pada penelitian ini akan dibahas pengembangan perangkat lunak \textit{routing engine} untuk Openstreetmap yang menyediakan fitur \textit{time-dependent} \textit{route-planning} dan \textit{online map-matching}. Penelitian ini juga membahas permasalahan, limitasi, serta peningkatan dari metode-metode pada penelitian sebelumnya.


