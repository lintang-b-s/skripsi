Penggunaan metode perhitungan rute terpendek pada perangkat lunak sumber terbuka hanya diterapkan pada jaringan jalan \textit{time-independent} dan tidak mengintegrasikan data prediksi dan historis kecepatan lalu-lintas, sehingga rute yang dihasilkan tidak mempertimbangkan kondisi lalu lintas. Solusi \textit{map-matching} yang terdapat pada perangkat lunak sumber terbuka hanya bisa dilakukan secara offline, yaitu trajektori GPS harus tersedia secara lengkap sebelum melakukan \textit{map-matching}. Penelitian ini berusaha mengatasi permasalah tersebut dengan mengembangkan perangkat lunak routing engine sumber terbuka yang menyediakan fitur \textit{time-dependent route planning} dan \textit{online map-matching}, dan mengintegrasikan data prediksi dan historis kecepatan lalu lintas ke perhitungan rute terpendek.
