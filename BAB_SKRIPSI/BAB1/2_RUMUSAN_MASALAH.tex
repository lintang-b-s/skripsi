Penggunaan algoritma dan teknik akselerasi dalam beberapa perangkat lunak \textit{routing engine} terbukti mampu mempercepat proses pencarian rute terpendek hingga beberapa kali lipat dibandingkan dengan algoritma Dijkstra standar. Namun, sebagian besar metode tersebut masih diterapkan pada jaringan jalan time-independent, yaitu kondisi di mana bobot setiap sisi (misalnya waktu tempuh pada segmen jalan) dianggap konstan dan tidak berubah seiring waktu. Asumsi time-independent ini menyebabkan hasil rute terpendek yang diperoleh tidak mempertimbangkan variasi kondisi lalu lintas, baik berdasarkan prediksi maupun data historis. Akibatnya, rute yang dipilih berpotensi tidak optimal, terutama ketika dalam jangka waktu dekat terjadi peningkatan kepadatan lalu lintas atau kemacetan yang signifikan. Penggunaan algoritma dan metode untuk map matching yang diimplementasikan pada beberapa perangkan lunak sumber terbuka terbukti mampu menghasilkan trajektori lintasan yang akurat pada jaringan jalan, namun kebanyakan solusi yang ada hanya bisa dilakukan secara \textit{offline}, yaitu rute harus diselesaikan dulu sebelum bisa melakukan \textit{map-matching}, padahal kebanyakan aplikasi navigasi seperti Google Maps dan Apple Maps melakukan \textit{map-matching} secara \textit{real-time}. Untuk itu perlu adanya perangkat lunak \textit{routing engine} untuk mengatasi permasalahan diatas.
