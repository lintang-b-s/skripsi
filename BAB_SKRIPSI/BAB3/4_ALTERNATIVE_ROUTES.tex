Untuk mencari rute alternatif, \cite{Abraham2010} memperkenalkan metode heuristik \textit{admissible paths} Adimissble Paths untuk mencari rute alternatif pada jaringan jalan. Secara intuitif, rute alternatif berbeda secara signifikan dari rute tercepat $Opt$, tetapi rute tersebut masuk akal dan tidak lebih jauh daripada rute $Opt$ \cite{Abraham2010}. Diberikan semua kemungkinan rute $P$ dari simpul $s$ ke simpul $t$, rute $P$ disebut sebagai \textit{admissible alternative} jika memenuhi sifat-sifat berikut:
\begin{enumerate}
    \item \textit{Limited sharing}. Penjumlahan \textit{travel time} dari sisi-sisi yang muncul pada kedua rute \textit{Opt} dan rute \textit{P} ($\sigma(v)$) kurang dari $\gamma \cdot dist(s,t), 0 \leq \gamma \leq 1$.
    \item \textit{Local optimality}. Setiap subrute dari $P$ memiliki \textit{travel time} yang lebih kecil dari $\alpha \cdot dist(s,t), 0<\alpha<1$.
    \item \textit{Bounded stretch}. Setiap pasang simpul $u,v\in P$, berlaku $l(P_{uv})< (1+\epsilon)\cdot dist(u,v), \epsilon \geq 1$. 
\end{enumerate}

Agar metode \textit{admissible paths} bisa diintegrasikan dengan Time-Dependent Customizable Route Planning (TD-CRP), diperlukan sedikit modifikasi pada algoritma kueri. Karena metode \textit{admissible paths} menentukan kandidat rute-rute alternatif dari himpunan rute $s-v-t$, dimana $v$ adalah simpul yang dikunjungi oleh \textit{forward search} dan \textit{backward search}, kueri TD-CRP perlu dibuat \textit{bidirectional}. \textit{Upper} dan \textit{lower bound} dari \textit{minimum travel time } dari simpul $s$ ke $t$ didefinisikan dengan:

\begin{align}
    Max_{G}(s,t)&=\min \{ \sum_{i=1}^{k-1} \max f_i  \mid  \begin{aligned}[t] 
    &(v_1 \rightarrow_{f_1} \ldots \rightarrow_{f_{k-1}} v_k ) \\
    & \text{ adalah rute dari simpul } s \text{ ke simpul } t \} 
       \end{aligned}
    \\ 
    Min_{G}(s,t)&=\min \{ \sum_{i=1}^{k-1} \min f_i \mid \begin{aligned}[t]  
    &(v_1 \rightarrow_{f_1} \ldots \rightarrow_{f_{k-1}} v_k ) \\
    &\text{ adalah rute dari simpul } s \text{ ke simpul } t \} 
     \end{aligned} \\
    Min_{G}(s,t)&\leq TTP_G(s,t)(\tau)\leq Max_G(s,t) \text{ untuk semua } \tau\in \mathbb{R}
\end{align}



\begin{algorithm}
\caption{BDTDDijkstra menghitung kandidat \textit{via nodes} $X=\{\ v\in V| v \text{ is visited by forward and backward search} \}\subseteq V$. \textit{Backward search} menghitung label $[q[u], r[u]]=[Min_G(t,u),Max(t,u)]$ untuk semua $u$ yang dikunjungi \textit{backward search}.} 
\label{alg:bidirectional-td-dijkstra-crp}
\resizebox{\textwidth}{!}{%
\begin{minipage}{\textwidth}
\scriptsize
\begin{algorithmic}[1]
\setstretch{0.9}
    \Procedure{BDTDDijkstra}{$s,t:V, \tau_0: \mathbb{R}$}: $(Set, \mathbb{R}, Set, Set, Set, Set)$
    \State $\tau[u]=\infty, \ u\in v, \ \tau[s]=\tau_0$ \Comment{\textit{node time travel label}}
    \State $[q[u],r[u]]=[\infty, \infty], \ u\in V, \ [q[t], t[t]]=[0,0]$ \Comment{\textit{Lower bound } dan \textit{upper Bound minimum travel time label}}
    \State $p_s[u]=\perp, p_t[u]=\emptyset, \ u\in V $
    \State $B=\infty, \ d=t$ \Comment{arah pencarian (\textit{forwardSearch} atau \textit{backwardSearch})}
    \State $X=\emptyset$
    \State $Q_s=\{(s_, sEntryPos, \tau_0)\}, \ Q_t=\{(t, tEntryPos,0)\}$
    \State $overlayQ_s=\emptyset, \ overlayQ_t=\emptyset $
    \While{$(Q_S \neq \emptyset \textbf{ or } Q_t\neq \emptyset) \textbf{ and }  \min \{Q_s.min(), overlayQ_s.min()\} + \min \{Q_t.min(), overlayQ_t.min()\}\leq (1+\epsilon) B$}
        \State $minG = \min \{ Q_s, Q_t \}$
        \State $minOverlayG = \min \{ overlayQ_s, overlayQ_t \}$
        \If{$ minG < minOverlayG$}
            \If {$Q_{\neg d}\neq \emptyset $}
                \State $d=\neg d$ 
            \EndIf 
            \State $(u, uEntryPos)=Q_d.deleteMin()$
            \If{$B<\infty \textbf{ and } \tau[u] + q[u] \leq B $}
                \State $X= X \cup \{(u, \tau[u]+r[u])\}$ \Comment{simpul $u$ dikunjungi oleh \textit{forward search} dan \textit{backward search}}
            \EndIf 
    
            \State $B= \min \{B, \ \tau[u]+r[u]\}$
            \For {$(u \rightarrow_f v, turnCost_{uv}) \in E_d $}
                \If{$d=s$}
                    \State \Call{tdRelaxCRP}{$u\rightarrow_f,v,v,u,uEntryPos,\tau,p_s,Q_s,overlayQ_s,turnCost_{uv}$}
                \Else 
                    \State \Call{intervalRelaxCRP}{$u,v,[q[u]+min f, \ r[u]+max f], q,r,p_t, Q_t,overlayQ_t$}
                \EndIf 
            \EndFor 
        \Else 
            \If {$overlayQ_{\neg d}\neq \emptyset $}
                \State $d=\neg d$
            \EndIf 
            \State $u=overlayQ_d.deleteMin()$
            \If{$B<\infty \textbf{ and } \tau[u] + q[u] \leq B $}
                \State $X= X \cup \{(u,\tau[u]+r[u])\}$ \Comment{simpul $u$ dikunjungi oleh \textit{forward search} dan \textit{backward search}}
            \EndIf 
    
            
            \State $B= \min \{B, \ \tau[u]+r[u]\}$
            \State $l_{st}(u)=\min\{MSD(PV(s), PV(u)), MSD(PV(t), PV(u))\}$
            \For {$u \rightarrow_{f_{shortcut\_uv}} v \in E_{d,C^{l_{st}(u)}} $}
                \If{$d=s$}
                    \State \Call{tdRelaxOverlayCRP}{$u\rightarrow_{f_{shortcut\_uv}}v,v,u, uEntryPoint,\tau,p_s,Q_s,overlayQ_s$}
                \Else 
                    \State \Call{intervalRelaxCRP}{$u,v,uEntryPoint,[q[u]+min f_{shortcut\_uv}, \ r[u]+max f_{shortcut\_uv}], q,r,p_t, Q_t,overlayQ_t$}
                \EndIf 
            \EndFor 
        \EndIf 
        
        
    \EndWhile
    \State \textbf{return} $\mathbf{X}, B, p_s, p_t, \tau, r$
    \EndProcedure
\end{algorithmic}
\end{minipage}%
}
\end{algorithm}


\begin{algorithm}
\caption{intervalRelaxCRP memperbarui label $[q[v], r[v]]=[Min_G(t,v),Max_G(t,v)]$} 
\label{alg:intervalRelaxCRP}
\resizebox{\textwidth}{!}{%
\begin{minipage}{\textwidth}
\scriptsize
\begin{algorithmic}[1]
\setstretch{0.9}
    \Procedure{intervalRelaxCRP}{$u,v, uEntryPoint, \ [q_{new}, r_{new}], q,r, p, Q, overlayQ, turnCost_{uv}$}
    \If{$q_{new}>r[v]$} 
        \State \textbf{return} \Comment{nilai \textit{Lowerbound travel time} baru dari $v$ melebihi \textit{Upperbound travel time} lama dari $v$}
    \EndIf 
    \If{$r_{new}<q[v]$}
        \State $p[v]=\emptyset$ \Comment{nilai \textit{Upperbound travel time} baru dari $v$ kurang dari \textit{Lowerbound travel time} lama dari $v$}
    \EndIf
    \State $p[v]= u\cup p[v]$
    \If{$q_{new}\geq q[v] \textbf{ and }r_{new} \geq r[v]$}
        \State \textbf{return} \Comment{nilai \textit{lowerbound travel time} dan \textit{upperbound travel time} baru melebihi nilai lama}
    \EndIf 

    \State $l_{st}(v)=min\{MSD(PV(s), PV(v)), MSD(PV(t), PV(v))\}$
    \If{$l_{st}(v) =0$ }
        \State $[q[v], r[v]]=[\min\{q[v], q_{new}+turnCost_{uv}\}, \min \{r[v], r_{new}+turnCost_{uv}\}]$
        \If{$v\notin Q$}
            \State $Q.insert(v, q[v])$
        \Else 
            \State $Q.decreaseKey(v, q[v])$
        \EndIf 
    \Else 
        \State $vOverlay = GetOverlayVertex(v, uEntryPoint)$
        \State $[q[vOverlay], r[vOverlay]]=[\min\{q[vOverlay], q_{new}+turnCost_{uv}\}, \min \{r[vOverlay], r_{new}+turnCost_{uv}\}]$
        \If{$vOverlay\notin overlayQ$}
            \State $overlayQ.insert(vOverlay, q[vOverlay])$
        \Else 
            \State $overlayQ.decreaseKey(vOverlay, q[vOverlay])$
        \EndIf 
    \EndIf 
    \EndProcedure

    
\end{algorithmic}
\end{minipage}%
}
\end{algorithm}

Algoritma pencarian rute alternatif dimulai dari menjalankan $BDTDDijkstra$ untuk mendapatkan kandidat \textit{via nodes} $v\in V$. Algoritma $BDTDDijkstra$ terinspirasi dari kueri \textit{bidirectional} pada Time-Dependent Contraction Hierarchies yang diperkenalkan pada penelitian oleh \cite{Veit2013}. Kueri Bidirectional Time-Dependent Dijkstra untuk Customizable Route Plannning ditunjukkan pada Algoritma~\ref{alg:bidirectional-td-dijkstra-crp}. Dalam \textit{bidirectional search}, memperbarui \cite{Veit2013} \textit{upperbound} $B$, jika simpul $u$ yang baru saja di \textit{settled} sudah dikunjungi pada \textit{forward search} dan \textit{backward search}. $B$ adalah \textit{upperbound} dari $EA_G(s,t,\tau_0)$ dengan nilai awal $\infty$. \textit{Backward search} adalah aproksimasi dari $ProfileSearch$ dan menghitung label $[q[u], r[u]]=[Min_G(t,u), Max_G(t,u)]$ untuk semua simpul $u$ yang dikunjungi pada kedua \textit{search}. Simpul $v$ yang dikunjungi pada kedua \textit{search} termasuk dalam \textit{candidate/via nodes} $X$. \textit{Via node} $u$ adalah simpul $u$ yang memiliki $\tau[u] + r[u] < \infty$. \textit{Forward search} menjalankan Algoritma~\ref{alg:td-dijkstra-crp}. Algoritma $BDTDDijkstra$ menghitung \textit{upperbound} $B$ yang merupakan penjumlahan dari label \textit{minimum departure time} tentatif \textit{forward} search $\tau[u]\in \Pi$ dan \textit{upperbound minimum travel time} $r[u]\in \mathbb{R}$ dari \textit{backward search} dengan \textit{source} $t$, untuk semua simpul $u$ yang dikunjungi pada kedua pencarian. Dengan kata lain, untuk simpul-simpul $u$ yang termasuk ke dalam \textit{via nodes} $X$, kita bisa melakukan perjalanan dari $s$ ke $t$ melalui $u$ dengan mengunjungi sisi-sisi yang dikunjungi \textit{forward search} yang optimal (menghasilkan \textit{minimum travel time } untuk waktu keberangkatan $\tau_0$) dan mengunjungi sisi-sisi yang optimal (memiliki nilai minimal dari \textit{upper bound} TTF dari sisi) yang dikunjungi backward search. Untuk setiap \textit{via node} $v$, cari rute tercepat $s-v$ dan $v-t$ dengan menggunakan $TDDijkstraCRP$ yang jika digabung akan menghasilkan rute $P_v$ dengan \textit{time travel} $l(v)$. Untuk semua kemungkinan rute $P$ dari $s$ ke $t$, $P_v$ memiliki nilai \textit{stretch} yang sangat kecil, atau dengan kata lain $l(P_v)\leq l(P)$ (\cite{Abraham2010}). Hitung skor $f(v)=2l(v)+\sigma(v)-pl(v)$. Dimana $l(v)$ adalah \textit{travel time } tentatif optimal dari $v$ ( $(\tau[v]+r[v])-\tau_0$) saat menjalankan $BDTDDijkstra$. $\sigma(v)$ adalah penjumlahan \textit{travel time} dari sisi-sisi yang muncul pada kedua rute \textit{Opt} dan rute $P_v$. $pl(v)$ adalah plateau dari simpul $v$. Plateau $pl(v)$ dapat dicari dengan membuat subrute $P_{plateau\_v}\subseteq Opt$ yang berisi simpul-simpul $u\in P_{plateau\_v}$ dengan $dist(s,u)+dist(u,t)=l(v)$, $pl(v)$ dihitung dengan menjumlahkan bobot \textit{time travel} dari sisi-sisi $P_{plateau\_v}$. Untuk memastikan kandidat rute alternatif memenuhi kriteria \textit{admissible paths}, kita membuang kandidat yang memiliki: $l(v)\geq (1+\epsilon)\cdot l(Opt), \sigma(v)\geq\gamma \cdot l(Opt), \text{ atau } pl(v)\leq \alpha \cdot l(Opt)$. Rute-rute alternatif yang dikembalikan adalah kandidat-kandidat yang memiliki $f(v)$ terkecil. Algortima pencarian rute alternatif \textit{admissible paths} ditunjukkan pada Algoritma~\ref{alg:alternative-routes}. Algoritma ini memiliki kompleksitas waktu $O(C_v\cdot (E_1+U_1\log U_1+B^{2}+B \log B))$, dimana $C_v$ adalah jumlah \textit{via node} $v$ yang ditemukan oleh algoritma $BDTDDijkstra$.


\begin{algorithm}
\caption{findAlternativeRoutes mengembalikan sejumlah $k$ rute alternatif terbaik} 
\label{alg:alternative-routes}
\resizebox{\textwidth}{!}{%
\begin{minipage}{\textwidth}
\scriptsize
\begin{algorithmic}[1]
\setstretch{0.9}
    \Procedure{findAlternativeRoutes}{$s,t: V, \tau_0: \mathbb{R}, k: \mathbb{Z}^+$}: $Set$
        \State $(l_{Opt}, shortestPath, \cdot, \cdot) =$\Call{tdDijkstraCRP}{$s,t, \tau_0$} 
        \State $(\mathbf{X}, \cdot, p_{sbd}, p_{tbd}, \tau_{bd}, r)=$\Call{BDTDDijkstra}{$s,t, \tau_0$}
        \State $potentialCandidates = \emptyset$
        \Procedure{calculatePlateau}{$\tau_{sbd},\tau_{tbd},v, l_v, p_{s}, p_t, \tau_0$}: $\mathbb{R}$
             \State $plateau = 0$
             \State $u = v$
             \While{$u \neq \perp$} \Comment{backtrack}
                    \If{$u\in p_t \textbf{ and } (\tau_{sbd}[u]-\tau_0)+ \tau_{tbd}[u] = l_v $} \Comment{$dist(s,u)+dist(u,t)=l(v)$}
                        \State $cost_{uf}= (\tau_{sbd}[u]-\tau_0)-(\tau_{sbd}[p_{s}[u]]-\tau_0)$
                        \State $ plateau = plateau + cost_{uf} $
                        \State $u = p_{s}[u]$
                    \EndIf
             \EndWhile
             \State $u = v$
             \While{$u \neq \perp$} \Comment{backtrack}
                    \If{$u\in p_s \textbf{ and } (\tau_{sbd}[u]-\tau_0)+ \tau_{tbd}[u] = l_v $} \Comment{$dist(s,u)+dist(u,t)=l(v)$}
                        \State $cost_{ub}= \tau_{tbd}[u]-\tau_{tbd}[p_{t}[u]]$
                        \State $ plateau = plateau +  cost_{ub} $
                        \State $u = p_{t}[u]$
                    \EndIf
             \EndWhile
             \State \textbf{return} $plateau$
        \EndProcedure
    
        \Procedure{calculatedistanceShare}{$Opt, P$}: $\mathbb{R}$
            \State $distanceShare= 0$
            \For{$p\in P$}
                \If {$p \in Opt$}
                    \State $distanceShare = distanceShare + p.weight$ 
                \EndIf 
            \EndFor
            \State \textbf{return} $distanceShare$
        \EndProcedure
        \For {$candidate \in X$}
            \State $(svTimeTravel, shortestPath, p_s, \tau_{sv}) = $\Call{tdDijkstraCRP}{$s,candidate.node, \tau_0$} 
            \State $(vtTimeTravel, shortestPath, p_t, \tau_{vt}) = $\Call{tdDijkstraCRP}{$candidate.node, t, \tau_0$} 
            \State $l_v=svTimeTravel+vtTimeTravel$
            \If{$l_v \geq (1+\epsilon)\cdot l_{Opt} $}
                \State $\textbf{continue}$
            \EndIf

            
            \State $edgesSV=$\Call{backtrack}{$p_s, candiate.node$}
            \State $edgesVT=$\Call{backtrack}{$p_t, candiate.node$}
            \State $\sigma_v=$\Call{calculatedistanceShare}{$shortestPath, edgesSV \cup edgesVT$}
            \If{$\sigma_v\geq \gamma \cdot l_{Opt}$}
                \State \textbf{continue} 
            \EndIf
            \State $pl_v=$\Call{calculatePlateau}{$\tau_{bd},r,candidate.node,  l_v, p_{sbd}, p_{tbd}, \tau_0$}
            \If{$pl_v \leq \alpha \cdot l_{Opt}$}
                \State \textbf{continue} 
            \EndIf
            \State $f_v=2l_v+\sigma_v-pl_v$
            \State $potentialCandidates  = potentialCandidates \cup \{candidate.node\}$
        \EndFor
        \State $sortedCandidates=$\Call{sort}{$potentialCandidates$}
        \State \textbf{return} $sortedCandidates[:k]$
    \EndProcedure

    
\end{algorithmic}
\end{minipage}%
}
\end{algorithm}



