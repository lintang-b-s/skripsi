\subsection{Deskripsi Penelitian}
\label{subsec:research-description}
Tujuan dari penelitian ini adalah untuk membangun perangkat lunak sumber terbuka \textit{routing engine} untuk Openstreetmap yang mendukung fitur \textit{time-dependent route planning} yang diintegrasikan dengan model prediksi kecepatan lalu-lintas dan \textit{online map-matching}. Penulis berharap perangkat lunak \textit{routing-engine} yang dikembangkan dapat menghasilkan rute optimal, akurat, serta mampu menghindari kemacetan. Dengan fitur \textit{online map-matching} yang dikembangkan, penulis berharap pengguna perangkat lunak dapat memudahkan pengguna dalam melakukan navigasi jarak-jauh. Penulis juga berharap, perangkat lunak yang dikembangkan juga dapat dimanfaatkan oleh perusahaan yang sangat membutuhkan layanan navigasi pada operasional perusahaan mereka, sekaligus membantu menurunkan biaya operasional perusahaan dikarenakan perangkat lunak bersifat sumber terbuka.

Penelitian akan menggunakan \textit{dataset} kecepatan lalu lintas dari hasil \textit{web scraping} situs waze.com selama 30 hari (periode 22 Oktober 2025 - 22 November 2025). Data peta yang digunakan diambil dari data Openstreetmap dengan \textit{bounding box} koordinat geografis $(minLongitude,minLatitude,maxLongitude,maxLatitude)=(110.132,-8.2618,110.9221,-6.888)$ yang meliputi wilayah area Daerah Istimewa Yogyakarta, Klaten, Semarang, Salatiga, dan Surakarta. Data peta akan diproses oleh perangkat lunak dan dibuat struktur data graf dengan representasi \textit{adjacency array} \cite{Mehlhorn2008}. Setelah itu perangkat lunak akan menjalankan fase \textit{preprocessing} dari Customizable Route Planning yang dijelaskan pada subbab~\ref{subsec:tdcrp-preprocessing}. \textit{Web scraping} dilakukan setiap 20 detik dan data akan di aggregasi dan di rata-rata untuk interval 5 menit. \textit{Piecewise linear function} yang memiliki interval 20 menit untuk setiap hari Senin-Jumat akan dibuat dengan  merata-rata data trafik pada setiap minggu di hari tersebut. Setelah mendapatkan data kecepatan lalu lintas 30 hari, akan dilakukan \textit{training} model MTGNN. Setelah pelatihan, Setiap 20 detik akan dilakukan \textit{web scraping} data kecepatan lalu lintas dari situs waze.com, dan setiap 15 menit data akan di agregasi dan dirata-rata untuk interval 5 menit. Setiap 20 menit akan dilakukan prediksi kecepatan lalu-lintas satu jam kedepan menggunakan model MTGNN yang sudah dilatih. Data hasil prediksi dan \textit{piecewiese linear function} historis akan dijadikan sebagai input fase kustomisasi Time-dependent Customizable Route Planning yang dijelaskan pada subbab~\ref{subsec:tdcrp-kustomisasi}.

Fitur \textit{online map matching} akan di \textit{serve} dengan komunikasi protokol Websocket, yang mana \textit{client frontend} akan memanggil \textit{online map-matching} setiap koordinat GPS pengguna berubah. Evaluasi dari \textit{online map-matching} akan menggunakan data rute \textit{ground truth} yang penulis tentukan. Data \textit{test} menggunakan hasil dari \textit{online map-matching} yang akan dihasilkan oleh perangkat lunak dari data titik-titik GPS yang mengikuti rute \textit{ground truth}. Data titik-titik GPS dihasilkan oleh penulis dengan cara berkendara mengikuti rute \textit{ground truth} selagi menggunakan fitur \textit{online map-matching} dari perangkat lunak yang dikembangkan.
