Prediksi kecepatan lalu lintas merupakan komponen penting dalam sistem transportasi cerdas, karena hasil dari prediksi bisa digunakan untuk pengambilan keputusan yang lebih baik dalam perencanaan rute, navigasi \textit{real-time}, dan manajemen lalu lintas. Metode prediksi lalu lintas terbagi menjadi dua kategori, yaitu berbasis model dinamis dan metode berbasis data. Pendekatan dinamis memodelkan evolusi temporal kondisi lalu lintas dari waktu ke waktu, berdasarkan hukum fisika atau matematika yang mendasari arus lalu lintas dan melakukan simulasi perilaku lalu lintas. Namun, pendekatan ini sering kali tidak realistis karena membutuhkan daya komputasi tinggi untuk simulasi berskala besar (\cite{Vlahogianni2015}). Model statistik klasik dan pembelajaran mesin adalah contoh dari metode berbasis data. Model statistik klasik seperti ARIMA banyak digunakan untuk prediksi lalu-lintas jangka pendek, Akan tetapi, model jenis ini dibatasi oleh asumsi stasioneritas dari \textit{time sequences} dan gagal memanfaatkan korelasi spasial antar ruas jalan, sehingga kinerjanya menjadi kurang optimal dalam melakukan prediksi lalu lintas secara akurat (\cite{AhmedCook1979}).

Penelitian oleh \cite{Wang2016} mengusulkan metode \textit{deep learning} bernama Error-feedback Recurrent Convolutional Neural Network (ECRNN). Model ini terdiri dari lima \textit{network layer}: \textit{input layer}, \textit{convolution layer}, \textit{pooling layer}, \textit{error-feedback recurrent layer}, dan \textit{output layer}. \textit{Input layer} membangun matriks \textit{spatio-temporal} yang menyimpan kecepatan lalu lintas dari setiap segmen yang terhubung secara spasial dari waktu ke waktu, yang memungkinkan model untuk mempelajari ketergantungan antar segmen secara alami. Fungsi dari \textit{convolution layer} dan \textit{pooling layer} adalah untuk mengekstrak fitur dari matriks \textit{input spatio-temporal}. \textit{Error-feedback recurrent layer} berfungsi untuk mempelajari fitur temporal dan mengkompensasi error prediksi dengan menggunakan hasil prediksi periode sebelumnya. \textit{Output layer} berfungsi untuk menghasilkan prediksi kecepatan lalu lintas. Peneliti mengevaluasi model dari data kecepatan GPS taksi yang dikumpulkan dari 202 jalan raya di kota Beijing, data gps di sampel setiap 5 menit. Peneliti membandingkan model ERCNN dengan Auto Regression Integrated Moving Average (ARIMA), Support Vector Regression (SVR), Stacked auto Encoders (SAE), dan 1D Convolutional Neural Network (1D-CNN). Model ECRNN mengungguli semua metode lainnya di semua \textit{benchmark} dengan RMSE sekitar 6\% pada panjang interval 10 menit. Namun, pada metode ECRNN \textit{convolution layer} hanya mengekstrak fitur dari segmen-segmen jalan yang terhubung dalam jangkauan 1-\textit{hop}, sehingga model tidak mempelajari fitur spasial jaringan transportasi secara optimal. Penelitian oleh \cite{Zhongjian2018} mengusulkan model Look-up Convolution Recurrent Neural Network(LC-RNN) untuk mengatasi keterbatasan model ECRNN dalam mempelajari fitur spasial. Model LC-RNN secara eksplisit mempelajari dinamika spasial dari jaringan jalan lokal melalui \textit{look-up convolution layer}. Kemudian \textit{LSTM layer} mempelajari pola ketergantungan temporal jangka panjang dari output \textit{look-up convolution layer} dengan mempertimbangkan konteks spasial sekitarnya. Selain mempelajari tren \textit{spatio-temporal}, model juga mengekstrak informasi lain seperti pola periodik (harian dan mingguan) dan faktor konteks, seperti cuaca, hari libur, dll. \textit{Fusion layer} memadukan output dari \textit{context extraction layer} \textit{peridoc layer}, dan \textit{LSTM layer} untuk menghasilkan prediksi kecapatan lalu lintas. Peneliti mengevaluasi model menggunakan Beijing \textit{trajectory dataset} dan membandingkan performanya dengan model Support Vector Regression, H-ARIMA, LSTM, SAE, Graph Convolution, ST-ResNet, dan DCNN. Pada dataset Beijing, model LC-RNN mendapatkan RMSE sebesar 4.686 mengungguli semua model yang dibandingkan. 

Beberapa penelitian terbaru menggunakan arsitektur Graph Neural Network untuk memprediksi kecepatan lalu lintas. Penelitan oleh \cite{Yu2018} mengusulkan model Spatio-Temporal Graph Convolutional Networks untuk prediksi kecepatan lalu lintas dengan memodelkan jaringan lalu lintas sebagai graf. Model STGCN terdiri dari dua blok ST-Conv dan sebuah \textit{fully-connected output layer}. Setiap blok ST-Conv terdiri dari dua Temporal Gated-Convolutional Layer dan sebuah Spatial Graph-Convolutional Layer. Spatial Graph-Convolutional Layer teridiri dari operasi \textit{graph convolutions} yang diaproksimasi degnan menggunakan \textit{Chebyshev Polynomials Approximation} orde 2 mampu mengekstrak fitur spasial yang penting. Temporal Gated-Convolutional Layer tediri dari satu Gated CNN yang merupakan 1-D Convolution dengan menggunakan fungsi aktivasi berupa Gated Linear Unit (GLU) (\cite{Gehring2017}). Peneliti mengevaluasi model STGCN dengan membandingkannya dengan Historical Average, ARIMA, LSVR, Feed-forward Neural Network, FC-LSTM, dan Graph Convolution GRU (GCGRU). Hasil eksperimen menunjukkan bahwa model STGCN menghasilkan performa terbaik untuk semua panjang horison prediksi dengan skor RMSE sebesar 5.20 pada panjang horison prediksi 15 menit (\cite{Yu2018}). Namun, prediksi model STGCN untuk banyak \textit{time step} berikutnya sangatlah lambat karena sifat model yang \textit{autoregressive} dimana prediksi pada \textit{time step} berikutnya berdasarkan prediksi \textit{time step} sebelumnya. Penelitian oleh \cite{Wu2019} mengusulkan model Graph Wavenet untuk mengatasi masalah lambatnya waktu prediksi STGCN untuk beberapa \textit{time step}. Banyak model Graph Neural Network sebelumnya mengasumsikan struktur graf lalu lintas tetap dan diketahui, padahal hubungan spasial yang sebenarnya seringkali tidak sepenuhnya tercermin dalam adjacency matrix yang ditentukan secara eksplisit. Untuk mengatasi hal ini, Graph Wavenet memperkenalkan \textit{self-adaptive adjacency matrix} dan \textit{stacked dilated causal convolution}. \textit{Self-adaptive adjacency matrix} mempelajari ketergantungan spasial tersembunyi tanpa pengetahuan mengenai \textit{adjacency matrix} yang sebenarnya yang dipelajari melalui \textit{stochastic gradient descent}. \textit{Stacked dilated causal convolution} memungkinkan \textit{receptive field} yang membesar secara eksponensial seiring dengan kedalaman \textit{layer}, hal ini memungkinkan jaringan konvolusi untuk menangkap sekuens yang lebih panjang dan menghemat sumber daya komputasi (\cite{Wu2019}). Model Graph Wavenet dievaluasi pada dua dataset publik: METR-LA (207 sensor lalu lintas di Los Angeles) dan PEMS-BAY (325 sensor di Bay Area). Dalam horizon prediksi 15, 30, dan 60 menit, Graph WaveNet secara konsisten mencapai performa terbaik dibandingkan dengan model ARIMA, FC-LSTM, WaveNet, DCRNN, GGRU, dan STGCN. Sebagai contoh, pada dataset METR-LA, Graph WaveNet memperoleh MAE 3.53, melampaui STGCN dengan MAE 4.59 dan dengan MAE DCRNN 3.60 untuk horison waktu 15 menit.

Tantangan utama dalam prediksi lalu lintas adalah kompleksitas hubungan spasial dan temporal yang saling memengaruhi antar lokasi serta keanekaragaman pola tersebut di berbagai wilayah (misalnya distrik bisnis dan perumahan memiliki pola lalu lintas yang berbeda). Sebagian besar model deep learning sebelumnya, seperti STGCN (\cite{Yu2018}) dan Graph Wavenet (\cite{Wu2019}), mengasumsikan bahwa ketergantungan spasial dan temporal bersifat seragam di seluruh node jaringan jalan. Penelitian oleh \cite{Pan2019} mengusulkan model ST-Metanet untuk mengatasi masalah ini dengan menggunakan meta-learning untuk menghasilkan parameter model berdasarkan atribut geografis tiap lokasi dan hubungan antar lokasi. Arsitektur ST-MetaNet berbasis pada Sequence-to-Sequence (Seq2Seq) yang terdiri dari \textit{encoder} dan \textit{decoder}. \textit{Encoder} berfungsi untuk mengkodekan sekuens informasi historis dari trafik $\{X_{t-\tau_{in}+1},\ldots,X_t\}$ menghasilkan \textit{hidden states} $\{H_{RNN}, H_{Meta-RNN}\}$, yang mana digunakan sebagai input dari \textit{decoder} yang selanjutnya memprediksi sekuens $\{\hat{Y}_{t+1},\ldots, \hat{Y}_{t+\tau_{out}}\}$. Baik encoder maupun decoder memiliki tiga komponen utama: (1) Meta-Knowledge Learner (NMK dan EMK) untuk mengekstrak representasi node dan edge dari atribut geografis seperti POI, GPS, dan kepadatan jaringan jalan, (2) Meta-GAT (Meta Graph Attention Network) untuk mempelajari korelasi spasial yang beragam melalui pembuatan parameter atensi yang berbeda untuk setiap pasangan node berdasarkan meta-knowledge, dan (3) Meta-RNN untuk memodelkan korelasi temporal yang beragam dengan menghasilkan bobot GRU yang spesifik untuk tiap node, sehingga setiap lokasi memiliki model temporalnya sendiri. ST-MetaNet diuji pada prediksi kecepatan lalu lintas menggunakan data lintasan GPS taksi di Beijing . Hasil eksperimen menunjukkan bahwa ST-MetaNet secara konsisten mengungguli model-model baseline seperti ARIMA, GBRT, Seq2Seq, GAT-Seq2Seq, ST-ResNet, dan DCRNN dalam metrik MAE dan RMSE. Model ST-MetaNet mendapatkan RMSE dan MAE secara berturut-turut sebesar (7.52,3.60), sedangkan Gat-SeqSeq (8.03,3.93) dan Seq2Seq (8.88, 4.38) untuk horison waktu 15 menit.

Penelitian oleh \cite{Guo2019} mengusulkan Attention based Spatial-Temporal Graph Convolutional Network (ASTGCN), sebuah model yang dirancang untuk meningkatkan akurasi prediksi aliran dan kecepatan lalu lintas. ASTGCN memadukan Spatial-Temporal Attention dan Graph Convolution pada dimensi temporal dan spasial, dan terstruktur menjadi tiga komponen independen untuk menangkap pola terkini, pola harian-periodik, dan pola mingguan-periodik. Input deret waktu dari ASTGCN dipecah menjadi tiga bagian, yaitu deret waktu terkni, deret waktu periode harian, dan deret waktu periode Mingguan. Masing-masing deret waktu menjadi input untuk masing masing tiga blok Spatio-Temporal. Spatial-Temporal Attention berfungsi untuk menangkap korelasi spasial dan temporal pada jaringan lalu lintas. Modul Graph Convolution yang diusulkan terdiri dari Graph Convolution dalam dimensi spasial, menangkap ketergantungan spasial dari lingkungan dan Graph Convolution sepanjang dimensi temporal, mengeksploitasi ketergantungan temporal dari waktu terdekat. Keluaran dari tiga blok komponen (terkini, harian, mingguan) digabungkan melalui bobot yang dapat dipelajari untuk menghasilkan prediksi akhir, yang memungkinkan model untuk secara adaptif menekankan pola temporal yang berbeda untuk segmen jalan yang berbeda. Untuk evaluasi, ASTGCN diuji pada dua \textit{dataset}, yaitu PeMSD4 dan PeMSD8. Model ini dibandingkan dengan delapan model, yaitu HA, ARIMA, VAR, LSTM, GRU, STGCN, GLU-STGCN, dan GeoMAN.Hasilnya menunjukkan bahwa ASTGCN mencapai RMSE dan MAE terendah pada kedua set data, mengungguli semua model. Sebagai contoh, pada data PEMS4, ASTGCN mendapatkan RMSE sebesar 32,82 mengungguli STGCN dengan RMSE 38,29 dan GeoMAN dengan RMSE 37,84. Metode berbasis GCN membutuhkan pra-pendefinisian interkoneksi graf dengan ukuran jarak, graf yang dihasilkan dengan cara ini mungkin mengandung bias dan tidak dapat diadaptasi ke domain tanpa pengetahuan mengenai jaringan jalan dari graf. Penelitian oleh \cite{Bai2020} mengusulkan model Adaptive Graph Convolutional Recurrent Network (AGCRN) yang dirancang untuk memodelkan ketergantungan spasial dan temporal tanpa bergantung pada graf yang sudah ditentukan sebelumnya. AGCRN terdiri dari dua modul: (1) modul Node Adaptive Parameter Learning (NAPL) untuk mempelajari pola spesifik dari simpul graf untuk setiap deret waktu, modul NAPL memfaktorkan parameter GCN ke dalam \textit{weights pool} dan \textit{bias pool}, yang memungkinkan setiap node lalu lintas memiliki parameter spesifiknya sendiri; (2) modul Data Adaptive Graph Generation (DAGG) untuk mempelajari \textit{node embedding } dari data dan menghasilkan graf secara adaptif selama pelatihan (\cite{Bai2020}). Peneliti menguji AGCRN pada \textit{dataset} PEMSD4 dan PEMSD8 dengan metrik MAE,RMSE, dan MAPE. Peneliti juga membandingkan AGCRN dengan baseline yang banyak digunakan dan beberapa model terbaik untuk prediksi kecepatan lalu lintas. Pada PEMSD8, AGCRN mendapatkan peingkatan sebesar 4-7\% terhadap model terbaik sebelumnya di seluruh metrik dan peningkatan sebesar 3-5\% untuk \textit{dataset} PEMSD4. Detail lengkap dari setiap dataset, metode, dan hasil penelitian yang ditinjau dapat dilihat pada Tabel~\ref{tab2:traffic_speed_forecasting} yang menunjukkan perbandingan seluruh penelitian sebelumnya.

