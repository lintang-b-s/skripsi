Berbagai penelitian sudah dilakukan untuk menyelesaikan masalah \textit{map matching} dengan berbagai metode. Penelitian yang dilakukan oleh \cite{Krumm2009} menggunakan Hidden Markov Model untuk menghasilkan urutan segmen jalan (\textit{hidden state}) dari pengukuran GPS yang memiliki \textit{noise} (\textit{observation}. Secara formal, \textit{hidden states} dari HMM adalah $N_r$ segmen-segmen jalan, $r_ i$, $i=1,\ldots, N_r$. Untuk setiap lokasi GPS $z_t$ dalam bentuk $(longitude,latitude)$, tujuannya adalah untuk menemukan segmen jalan tempat kendaraan sebenarnya berada. Diberikan observasi $z_t$, ada probabilitas emisi untuk setiap segmen jalan $r_i$, $p(z_t\mid r_i)$, yang memberikan \textit{likelihood}. Hal ini memberikan \textit{likelihood} bahwa observasi $z_t$ akan diamati jika kendaraan benar-benar berada di ruas jalan $r_i$. Secara intuitif, \textit{states} yang lebih dekat dengan lokasi observasi memiliki probabilitas emisi yang lebih tinggi dibdandingkan dengan \textit{states} yang jauh dari observasi. Jarak antara \textit{hidden state} dengan lokasi observasi adalah \textit{measurement error} yang diasumsikan mengikuti distribusi \textit{gaussian} dengan rata-rata nol, yaitu $p(z_t \mid r_i) = \frac{1}{2\pi\sigma_z^2} 
\exp\left( -\frac{1}{2} \frac{\lVert z_t - x_{t,i} \rVert^2}{\sigma_z^2} \right)$, di mana $\lVert z_t - x_{t,i} \rVert^2$ adalah \textit{haversine distance} dari titik observasi dengan titik proyeksi observasi ke kandidat segmen jalan. Probabilitas transisi memberikan probabilitas kendaraan bergerak di antara kandidat jalan yang cocok di antara observasi $z_t$ dan $z_{t+1}$. Intuisi dari probabilitas transisi adalah ketika jarak rute (yang dihasilkan algoritma \textit{dijkstra}) antara dua titik proyeksi ke kandidat segmen lebih besar dari jarak \textit{haversine} antara dua titik observasi, maka transisi sangat tidak mungkin, namun jika selisihnya sangat kecil maka transisi pada segmen jalan tersebut cocok. Bentuk histogram dari selisih jarak rute terpendek (yang dihasilkan dari algoritma \textit{dijkstra}) dari dua titik proyeksi observasi ke kandidat segmen jalan dengan jarak \textit{haversine} antara dua titik observasi $|\lVert x_{t,i} - x_{t+1,j} \rVert_{rute} - \lVert z_{t} - z_{t+1} \rVert_{haversine} |$ sangat mirip dengan kurva distribusi eksponensial. Sehingga, probabilitas transisi dimodelkan sebagai $p(d_t)=\frac{1}{\beta}e^{\frac{-|\lVert x_{t,i} - x_{t+1,j} \rVert_{rute} - \lVert z_{t} - z_{t+1} \rVert_{haversine}  |}{\beta}}$. Tujuan utama dari HMM \textit{map matching} adalah mencari $r^{*}_{1:T} = \arg\max_{r_{1:T}} \; \log \  p(r_{1:T} \mid z_{1:T})\underset{\text{bayes}}{=} \arg\max_{r_{1:T}} \; \log  \ \pi_1({r_1}) + \log \  \ p(z_1\mid r_1)+ \sum_{t=2}^{T} \left[ \log p(r_{t-1}, r_t) + \log \ p(z_t\mid r_t) \right] $ yang mana dapat dihitung dengan efisien dengan algoritma \textit{viterbi}  (\cite{Murphy2023}). Model ini mendapatkan akurasi sebesar 98\% dengan \textit{sampling interval} 2 detik pada \textit{dataset} data mengemudi mobil yang dikumpulkan di Nagakute, Jepang. Penelitian oleh \cite{Wei2012} meningkatkan akurasi dan kecepatan pencocokan dengan memodifikasi fungsi bobot dan algoritma \textit{map-matching}. Pada paper ini, fungsi bobot yang digunakan adalah $r^{*}_{1:T} = \arg\min_{r_{1:T}} \sum_{t=2}^2 t_t\lVert z_t - x_{t,i} \rVert + \frac{2\sigma^2 |\lVert x_{t,i} - x_{t+1,j} \rVert_{rute} - \lVert z_{t} - z_{t+1} \rVert_{haversine} |}{\beta} $, di mana $t_t$ adalah interval waktu antara $z_t$ dan $z_{t+1}$. Peneliti menggunakan $\text{R}^*-\text{tree}$ mencari kandidat segmen jalan terdekat dengan radius pencarian 30 meter. Algoritma \textit{dijkstra} mengkalkulasi rute terpendek dari setiap kandidat $z_t$ ke semua kandidat dari $z_{t+1}$, dengan membatasi pencarian dari titik asal dengan radius pencarian sebesar $50m/s \times t_t$. Hasil \textit{map-matching} dengan menggunakan teknik akselerasi ini mendapatkan akurasi sebesar 98.9\% dan hanya membutuhkan waktu 1.5 detik untuk \textit{map-matching} sebanyak 14,436 titik observasi.

Metode-metode yang dijelaskan diatas hanya bisa berlaku untuk \textit{offline map-matching}, dimana kita harus memiliki seluruh data lintasan gps untuk bisa melakukan \textit{map-matching}. Namun, aplikasi navigasi membutuhkan online map matching, yaitu algoritma yang dapat memperbarui hasil \textit{map-matching} secara \textit{real-time} saat titik GPS baru tiba. Pada penelitian yang dilakukan oleh \cite{Goh2012} mengembangkan \textit{Online Hidden Markov Model} (OHMM) untuk mengatasi masalah ini. Probabilitas emisi dimodelkan sebagai Gaussian, dengan tambahan faktor setengah dari lebar jalan $w=0.5 \cdot r.w$ dan fungsi penalti kecepatan untuk membedakan jalan paralel dengan batas kecepatan berbeda. Fungsi penalti kecepatan didefinisikan sebagai $S(v_t,v_r)=\frac{v_r}{\max (0,v_t-v_r)+v_r}$, dimana $v_t$ adalah kecepatan dari observasi $t$ dan $v_r$ adalah batas kecepatan maksimum dari segmen jalan $r (r.v)$. Probabilitas emisi didefinisikan sebagai $p(t\mid r)=S(v_t,v_r) \cdot \frac{1}{2w} \int_{-w}^{w} \frac{1}{\sqrt{2\pi}\sigma_g} \exp\left( -\frac{(l - d)^2}{2\sigma_g^2} \right) \, dl.$. Untuk mendifinisikan probabilitas transisi, OHMM mendifinisikan dua fungsi skor, yaitu fungsi diskrepansi jarak $T$ dan fungsi perubahan momentum $M$. Misalkan $i,j$ adalah pasangan kandidat segmen jalan yang dikaitkan dengan dua titik observasi berurutan. Fungsi diskrepansi jarak mengukur perbedaan antara jarak tempuh  $i\rightarrow j$  yang disimpulkan sensor dan panjang lintasan $i\rightarrow j$ yang diinterpolasi. Fungsi perubahan momentum $M$ mengukur perubahan momentum rata-rata yang dialami oleh kendaraan untuk setiap segmen jalan yang diambil dalam jalur $P_{i\rightarrow j}$. OHMM menggunakan Support Vector Machine (SVM) \textit{classifier} untuk mengklasifikasikan transisi sebagai benar atau salah, menghasilkan fungsi probabilitas transisi $p(i\rightarrow j)$, di mana vektor fitur terdiri dari skor komponen yang diberikan oleh fungsi skor $T$ dan $M$. OHMM menggunakan algoritma \textit{online viterbi} dengan \textit{variable sliding window} (VSW). \textit{Sliding window} ini meluas saat titik GPS baru diterima, dan mengecil saat ditemukan convergence point pada rantai Markov—yaitu titik di mana semua jalur yang bertahan di masa depan akan berisi sub-jalur yang sama, sehingga keputusan dapat diambil tanpa menunggu titik masa depan. Eksperimen pada data bus di Singapura menunjukkan bahwa OHMM dengan VSW mencapai akurasi di atas 90\% untuk interval \textit{sampling} kurang dari 1 menit (\cite{Goh2012}). Penelitian yang dilakukan oleh \cite{Liang2016} memperkenalkan metode Online Learning Hidden Markov Model (OLMM) untuk meningkatkan akurasi \textit{map-matching} pada kondisi lalu-lintas perkotaan yang berubah-ubah tanpa perlu melakukan \textit{training ulang} dengan data berlabel. OLMM memperbarui nilai paramater $\sigma$ dari probabilitas emisi secara adaptif dengan teknik \textit{online learning}. Nilai awal $\sigma=1$ untuk paramater awal, lalu kita menyimpan $m$ hasil \textit{matching} terakhir dan menghitung ulang $\sigma$ dengan persamaan $\sigma_z=1.4826MAD(\lVert P_t-P_i\rVert_{gc})$. Untuk menghitung probabilitas transisi, pertama kita harus menghitung  matriks \textit{n-connection} $a$ , yang nilai setiap elemen baris $i$ kolom $j$ adalah jumlah lompatan yang dibutuhkan oleh segmen jalan $i$ untuk dapat sampai ke segmen jalan $j$. Probabilitas transisi dapat kita hitung dengan persamaan berikut $a_{ij}=\frac{\frac{1}{a[i][j]}}{\sum_{k}\frac{1}{a[i][k]}} \text{, jika } a[i][j] \neq 0 \text{ dan } a_{ij}=0 \text{ ,jika } a[i][j] = 0$. Definisi dari probabilitas emisi OLMM sama dengan definisi probabilitas emisi HMM (\cite{Krumm2009}). Hasil eksperimen pada data lintasan gps pada peta Seattle menunjukkan bahwa OLMM mencapai akurasi sebesar 98.57\% dan stabil ketika ukuran jendela lebih dari 15 titik observasi (\cite{Liang2016}).


Penelitian oleh \cite{Jagadeesh2017} memperluas HMM untuk data gps yang \textit{sparse} (interval \textit{sampling} sangat rendah) dengan menggabungkannya dengan \textit{route choice model} (RCM). Definisi probabilitas emisi dari penelitan ini masih sama dengan definisi probabilitas emisi pada penelitian oleh (\cite{Krumm2009}). Untuk mendefinisikan probabilitas transisi, peneliti mengusulkan ukuran dari keberlikuan untuk jalur optimal $y(s_{t-1}, s_{t,k})$ dan ketidakmungkinan temporal $z(s_{t-1}, s_{t,k})$ antara \textit{state} $s_{t-1} \text{ dan } s_{t,k}$. Ukuran keberlikuan didefinisikan dengan $y(s_{t-1}, s_{t,k})=\frac{d(s_{t-1,j}. s_{t,k})-g(s_{t-1,j},s_{t,k})}{\Delta T}$, dimana $d$ adalah jarak rute terpendek dan g adalah jarak \textit{haversine}. Ketidakmungkinan temporal digunakan untuk menyaring  jalur yang tidak dapat dilalui dalam interval waktu T kecuali kendaraan berjalan dengan kecepatan yang sangat tinggi. Ketidakmungkinan temporal didefinisikan dengan $z(s_{t-1,j}, s_{t,k})=\frac{max((f(s_{t,1,j}, s_{t,k}) - \Delta T), 0)}{\Delta T}$, dimana $f(s_{t,1,j}, s_{t,k}) $ adalah waktu tempuh tercepat antara \textit{state} $s_{t-1,j} \text{ dan } s_{t,k}$. Peneliti berasumsi distribusi dari ketidakmungkinan temporal dan ukuran keberlikuan mengikuti distribusi eksponensial. Probabilitas transisi didefinisikan dengan $P(s_{t,k}\mid s_{t-1,j})=\lambda_y e^{-\lambda_y y(s_{t-1}, s_{t,k})} \lambda_z e^{-\lambda_z z(s_{t-1}, s_{t,k})}$. Algoritma \textit{online viterbi} digunakan untuk menghasilkan lintasan parsial yang mana akan dievaluasi ulang dengan beberapa rute alternative yang dihasilkan oleh \textit{choice set generation}. Peneliti juga mendefinisikan probabilitas pilihan rute, yang memberikan probabilitas dari pengemudi memilih rute alternative $p_i$ dari himpunan rute pilihan $C$ berdasarkan model logit multinomial yang diturunkan dari data rute aktual pengemudi. Untuk setiap rute alternative dan rute yang dihasilkan algoritma \textit{online viterbi}, hitung probabilitas \textit{route choice} dan probabilitas \textit{observation generation}, rute yang dihasilkan adalah rute dengan hasil perkalian probabilitas pilihan rute dan \textit{observation generation} tertinggi. Probabilitas \textit{observation generation} adalah probabilitas bahwa data lintasan observasi gps dihasilkan saat pengemudi melintasi di sepanjang jalan pada rute alternatif. Metode gabungan \textit{Online} HMM+RCM menghasilkan akurasi 91.3\% pada data \textit{sparse} dengan interval \textit{sampling} 1-5 menit.

\cite{Huang2022} mengusulkan metode Incremental Map Matching (IMM) yang dapat mengatasi lintasan GPS dengan interval \textit{sampling} yang sangat tinggi secara \textit{real-time} dan lintasan gps dicocokan secara \textit{batch} sehingga waktu komputasi jauh lebih cepat. Algoritma ini dimulai dengan memulai dari \textit{naive matching}, yaitu mencocokan titik=titik observasi gps ke titik proyekesi pada segmen jalan terdekat. Lalu, pada setiap titik gps yang baru diproses algoritma menghitung ukuran \textit{batch} $\delta$ dengan mempertimbangkan kecepatan saat ini, kecepatan rata-rata, interval \textit{sampling}, percepatan, dan panjang segmen jalan berikutnya. Setelah menghitung ukuran \textit{batch} $\delta$, jika algoritma mendeteksi bahwa $r(p_{i+\delta}) \neq r(p_i)$ dan $r(p_{i+\delta}) \neq r(p_{i+1})$, akan dijalankan strategi \textit{voting} untuk pencocokan segmen jalan untuk semua lintasan gps $p_{i+1}\rightarrow,\ldots,\rightarrow p_{i+\delta}$. Strategi voting memilih segmen jalan dengan \textit{voting} tertinggi untuk sublintasan gps $p_{i+k}, 1\leq k \leq \delta$. Hasil dari \textit{map-matching} untuk sublintasan $p_{i+k}, 1\leq k \leq \delta$ adalah segmen jalan dari hasil strategi \textit{voting}. Metode ini mendapatkan akurasi sebesar 92\% untuk interval \textit{sampling} 1 detik. Penelitian oleh \cite{Hu2023} mampu mengkalibrasi data observasi secara adaptif dan meningkatkan akurasi dalam berbagai kondisi lalu lintas yang kompleks. Setiap titik gps baru $g^{(i)}$ tiba, algoritma menghasilkan himpunan titik kandidat dengan \textit{K Nearest Neighbors} dengan radius pencarian $r$. Hitung skor fitur kecepatan $s_u(v_m^{(i)})$, posisi  $s_z(x_m^{(i)})$, dan arah $s_\theta(\alpha_m^{(i)})$ dari titik gps $i$ dengan mempertimbangkan fitur-fitur dari setiap titik kandidat, buang titik-titik kandiddat dengan $s_u(v_m^{(i)})=0$. Lalu, hitung $score^{(i)}$, vektor bobot $\mathbf{w}^{(i)}$, dan vektor probabilitas titik-titik kandidat $\mathbf{p}^{(i)}$ dengan persamaan $\mathbf{w}^{(i)*},\mathbf{p}^{(i)*}=\arg\max_{\mathbf{w}^{(i)},\, \mathbf{p}^{(i)}} \mathbf{w}^{(i)T} \mathbf{s}^{(i)} \mathbf{p}^{(i)}$, dimana matriks $s^{(i)} \in \mathbb{R}^{3\times |C^{(i)}|}$ adalah matriks skor fitur posisi, kecepatan, dan arah dan $\mathbf{w}$ adalah vektor bobot untuk setiap fitur. Jika $score^{(i)} < \tau$, hapus titik gps $g^{(i)}$ dan hentika proses pencocokan. Setelah mendapatkan skor, hasilkan kandidat dan rute optimal yang cocok dengan persamaan $c^{(i)*}=\min_{path}(length-\eta\sum_i p_{matched}^{(i)})$. Jika titik gps saat ini dan titik gps sebelumnya dihapus, maka jalankan mekanisme koreksi retrospektif dimana algoritma melakukan koreksi kembali terhadap hasil matching sebelumnya sehingga mengurangi kesalahan detour. Hasil eksperimen pada \textit{dataset} peta Shanghai dan Singapura menunjukkan bahwa AMM mengalami peningkatan akurasi sebesar 32\% dibandingkan dengan metode HMM (\cite{Krumm2009}) pada kondisi lalu lintas yang kompleks (\cite{Hu2023}).

Meskipun metode-metode \textit{map matching} terdahulu seperti HMM dan variasinya memberikan akurasi yang tinggi, namun hasil dari pendekatan-pendekatan \textit{online map-matching} yang dijelaskan diatas adalah lintasan parsial yang sudah di cocokkan dari seluruh lintasan gps. Dengan kata lain, hasil \textit{online map-matching} mengalami keterlambatan dalam menentukan posisi kendaraan terkini. Hal ini menyebabkan metode-metode \textit{ online map matching} diatas tidak dapat memberikan respons secara \textit{real-time}, yang menjadikannya tidak cocok untuk digunakan dalam aplikasi navigasi. Hasil penelitian \textit{map-matching} yang ditinjau dapat dilihat pada tabel 2.2 yang menunjukkan perbandingna seluruh penelitian sebelumnya.


